\documentclass[12pt]{article}
\title{\centering{Introduction to Git}}
\author{Hammad Shaikh}

\begin{document}

\maketitle

\section{Motivation for using Git}
Git is a program that allows you to keep track of several versions of your coding files. It keeps a history of all the changes and you can revert to any point in the history. Git can be used locally on your own computer, or you can also host a remote repository that can be used to collaborate with others and keep back ups of your script files.

You actually don't need to know a lot about coding to use Git if use a GUI such as Source Tree or GitKraken. 

\section{Getting started with Git}
We can get started with Git by installing the program, configuring the settings, and initializing a new folder with Git. In this tutorial we will get started with tracking versions of a TeX file using Git.

\section{Using Git without any code}
You don't necessarily need to be an expert in the Git terminal commands to do version control for your projects. There are graphical user interfaces available for Git that you can use to add commits, push to remote repository, create branches, etc using just a few clicks. The following Git GUI are fairly popular 1)  Sourcetree, 2) GitKraken, and 3) GitHub desktop.


\end{document}
